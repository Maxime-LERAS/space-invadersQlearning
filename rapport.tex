\documentclass[12pt,oneside,a4paper]{article}
\usepackage[table]{xcolor}
\usepackage{graphicx}
\usepackage{amsmath}
\usepackage{fancyhdr}

\begin{document}

\title{Rapport Q-Learning Space Invaders \\ G1I0M3 \\ Étude de l'impact de gamma sur les résultats}

\author{Julien Molinier, Maxime Leras}
\date{7 Mai 2020}
\maketitle\thispagestyle{empty}

\newpage    
\clearpage
\thispagestyle{empty}
\renewcommand*\contentsname{Sommaire}
\tableofcontents
\newpage

\pagestyle{fancy}
\cfoot{\thepage}
\fancyhead{}
\fancyhead[R]{\leftmark}

\pagenumbering{arabic}
\section{Introduction}
\paragraph{}
    Space Invaders est un jeu vidéo d'arcade sorti en 1978. Le principe
    est simple : détruire des vagues d'aliens se déplaçant horizontalement
    grâce à des tirs.
    Le Q-Learning est une technique d'apprentissage automatique
    interéssante pour ce type de jeu car elle ne nécessite aucun modèle 
    initial de l'environnement.
\paragraph{}
    Ce rapport a pour but d'étudier l'impact du paramètre $\gamma$ sur 
    les résultats de l'apprentissage (facteur d'actualisation qui détermine 
    l'importance des récompenses futures).

\section{Description de l'environnement de jeu}
\subsection{Description de l'état du jeu}
\subsection{Les récompenses}
\subsection{Traitement des données}


\section{Etude du paramètre $\gamma$}

\end{document}